\section{abstract}
Go is an ancient game that first appeared in China. It is popular especially in East Asian. The game of Go has long been viewed as the most challenging of classic games for artificial intelligence owing to its enoumous search space and the difficulty of evaluating board positions and moves. Due to the computing complexity, it has been long been considered that the computer won't be able to beat human in this game. Recent years have witnessed rapid development in the area of AI. Human are forced to change their belief about the things machine can do over time. An event that happened earlier this year challenged this belief.

Researchers at Google managed to develop the AI program the beat the chess champion of Europe. This is regarded as a great improvement considering the computing complexity of Go. Benefiting from the development of computing capability of computers and the application of machine learning technique, it seems that human intelligence are likely to be able to beaten by machine intelligence. 

Rome wasn't built in a day. This paper presents a few major methods and techniques which were come up with by intelligent people in the process to improve the intelligence of computers. There have been generally three stages in the process of trying to solve the problem of chess playing. The first is rule based and searching algorithm; the second stage is about Monte-Carlo Go~\cite{brugmann1993monte, enzenberger2010fuego}; the third stage is the application of deep neural network. Even in recent years, there have been different attempts to tackle this problem. However, some researchers also argues that there is some missing piece in AlphaGo. It is not essentially comparable to human wisdom. It's a specific technique to the field of game playing.

During the process of this project, the author read a series of papers in recent years on Go. This report is a summary of what the author has touched in this process.
